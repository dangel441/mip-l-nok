% Metódy inžinierskej práce

\documentclass[10pt,twoside,slovak,a4paper]{article}

\usepackage[slovak]{babel}
%\usepackage[T1]{fontenc}
\usepackage[IL2]{fontenc} % lepšia sadzba písmena Ľ než v T1
\usepackage[utf8]{inputenc}
\usepackage{graphicx}
\usepackage{wrapfig}
\usepackage{amsmath}
\usepackage{url} % príkaz \url na formátovanie URL
\usepackage{hyperref} % odkazy v texte budú aktívne (pri niektorých triedach dokumentov spôsobuje posun textu)

\usepackage{cite}
%\usepackage{times}

\pagestyle{headings}

\title{Inteligentné odporúčacie systémy v zdravotníctve\thanks{Semestrálny projekt v predmete Metódy inžinierskej práce, ak. rok 2024/25, vedenie: Ing. Ivan Kapustík}} % meno a priezvisko vyučujúceho na cvičeniach

\author{Daniel Vanek\\[2pt]
	{\small Slovenská technická univerzita v Bratislave}\\
	{\small Fakulta informatiky a informačných technológií}\\
	{\small \texttt{xvanek@stuba.sk}}
	}

\date{\small 30. september 2024} % upravte



\begin{document}

\maketitle

\begin{abstract}
V oblasti medicíny hrá veľkú rolu ľudský faktor. Lekárom nestačia len teoretické poznatky, nadobudnuté počas štúdia. Doktor najmä praxou získava skúsenosti a učí sa na svojich chybách. Postupom času, vyvíjaním nových technológií a zväčšovaním dostupnosti technológií pribúdajú faktory, ktoré treba pri rozhodnutiach zohľadniť. V súčasnej informačnej spoločnosti, v ktorej neustále narastá množstvo dostupných dát, prichádzajú na pomoc odporúčacie systémy. Tie vedia spracovať väčšie množstvo dát a v kratšom časovom úseku ako človek. Vďaka nim by sa mohlo znížiť množstvo chýb spojených s ľudským faktorom. Či už v diagnóze pacienta alebo predpisovaní liekov na základe dostupných informácii. Cieľom tohto článku je poskytnutie prehľadu možných aplikácii odporúčacích systémov v tejto sfére. V článku sa pokúsim priblížiť využitie odporúčacích systémov v zdravotníctve na jednej z najrozšírenejších chorôb – Diabetes. Je taktiež dôležité vytýčiť, že odporúčacie systémy neslúžia ako náhrada kvalifikovaného odborníka, ale len ako podpora rozhodovacieho procesu. Doktor sa tak následne môže utvrdiť vo svojom názore alebo získať iný prístup k problematike, ktorý môže zvážiť a za normálnych okolností by mu nemusel napadnúť.  Hovoríme tak o spolupráci umelej inteligencie a človeka, ktorá využíva ľudské prednosti ako intuíciu, morálnosť a zároveň ťaží z kapacít ktoré ponúka umelá inteligencia. Spolu s prehľadom takýchto systémov sa taktiež pokúsim predostrieť možné nástrahy pri využívaní odporúčacích systémov. 
\end{abstract}



\section{Úvod}
\includegraphics[scale=0.3]{Kolobeh_vody.png}
\includegraphics[scale=0.4]{Vznik_pisomky.png}
Motivujte čitateľa a vysvetlite, o čom píšete. Úvod sa väčšinou nedelí na časti.

Uveďte explicitne štruktúru článku. Tu je nejaký príklad.
Základný problém, ktorý bol naznačený v úvode, je podrobnejšie vysvetlený v časti~\ref{nejaka}.
Dôležité súvislosti sú uvedené v častiach~\ref{dolezita} a~\ref{dolezitejsia}.
Záverečné poznámky prináša časť~\ref{zaver}.



\section{Nejaká časť} \label{nejaka}

Z obr.~\ref{f:rozhod} je všetko jasné. 

\begin{figure*}[tbh]
\centering
%\includegraphics[scale=1.0]{diagram.pdf}
Aj text môže byť prezentovaný ako obrázok. Stane sa z neho označný plávajúci objekt. Po vytvorení diagramu zrušte znak \texttt{\%} pred príkazom \verb|\includegraphics| označte tento riadok ako komentár (tiež pomocou znaku \texttt{\%}).
\caption{Rozhodujúci argument.}
\label{f:rozhod}
\end{figure*}

\[
\begin{bmatrix}
11 & 12 & 13 & 14 \\
21 & 22 & 23 & 24 \\
31 & 31 & 33 & 34 \\
41 & 42 & 43 & 44 \\
51 & 52 & 53 & 54
\end{bmatrix}
\]



\section{Iná časť} \label{ina}

Základným problémom je teda\ldots{} Najprv sa pozrieme na nejaké vysvetlenie (časť~\ref{ina:nejake}), a potom na ešte nejaké (časť~\ref{ina:nejake}).\footnote{Niekedy môžete potrebovať aj poznámku pod čiarou.}

Môže sa zdať, že problém vlastne nejestvuje\cite{Coplien:MPD}, ale bolo dokázané, že to tak nie je~\cite{Czarnecki:Staged, Czarnecki:Progress}. Napriek tomu, aj dnes na webe narazíme na všelijaké pochybné názory\cite{PLP-Framework}. Dôležité veci možno \emph{zdôrazniť kurzívou}.


\subsection{Nejaké vysvetlenie} \label{ina:nejake}

Niekedy treba uviesť zoznam:


\begin{equation}
\sum_{i=1}^{n} i = \frac{n(n+1)}{2} =  \frac{1(2+1)}{2}+\frac{2(2+2)}{2}+\frac{3(2+3)}{2}+ ...
 + \frac{(n-1)(2+(n-1))}{2}+\frac{n(2+n)}{2}
\end{equation}


Ten istý zoznam, len číslovaný:

\begin{enumerate}
\item jedna vec
\item druhá vec
	\begin{enumerate}
	\item x
	\item y
	\end{enumerate}
\end{enumerate}


\subsection{Ešte nejaké vysvetlenie} \label{ina:este}

\paragraph{Veľmi dôležitá poznámka.}
Niekedy je potrebné nadpisom označiť odsek. Text pokračuje hneď za nadpisom.



\section{Dôležitá časť} \label{dolezita}




\section{Ešte dôležitejšia časť} \label{dolezitejsia}


\section{Doplnený odstavec}

\section{Záver} \label{zaver} % prípadne iný variant názvu
\begin{wrapfigure}{r}{0.4\textwidth}
  \centering
  \includegraphics[width=0.38\textwidth]{STU-FIIT-ncb.png}
\end{wrapfigure}


%\acknowledgement{Ak niekomu chcete poďakovať\ldots}


% týmto sa generuje zoznam literatúry z obsahu súboru literatura.bib podľa toho, na čo sa v článku odkazujete
\bibliography{literatura}
\bibliographystyle{abbrv} % prípadne alpha, abbrv alebo hociktorý iný
\end{document}
